% Options for packages loaded elsewhere
\PassOptionsToPackage{unicode}{hyperref}
\PassOptionsToPackage{hyphens}{url}
%
\documentclass[
  14pt,
]{article}
\usepackage{amsmath,amssymb}
\usepackage{lmodern}
\usepackage{iftex}
\ifPDFTeX
  \usepackage[T1]{fontenc}
  \usepackage[utf8]{inputenc}
  \usepackage{textcomp} % provide euro and other symbols
\else % if luatex or xetex
  \usepackage{unicode-math}
  \defaultfontfeatures{Scale=MatchLowercase}
  \defaultfontfeatures[\rmfamily]{Ligatures=TeX,Scale=1}
  \setmainfont[]{MerriWeather}
\fi
% Use upquote if available, for straight quotes in verbatim environments
\IfFileExists{upquote.sty}{\usepackage{upquote}}{}
\IfFileExists{microtype.sty}{% use microtype if available
  \usepackage[]{microtype}
  \UseMicrotypeSet[protrusion]{basicmath} % disable protrusion for tt fonts
}{}
\makeatletter
\@ifundefined{KOMAClassName}{% if non-KOMA class
  \IfFileExists{parskip.sty}{%
    \usepackage{parskip}
  }{% else
    \setlength{\parindent}{0pt}
    \setlength{\parskip}{6pt plus 2pt minus 1pt}}
}{% if KOMA class
  \KOMAoptions{parskip=half}}
\makeatother
\usepackage{xcolor}
\IfFileExists{xurl.sty}{\usepackage{xurl}}{} % add URL line breaks if available
\IfFileExists{bookmark.sty}{\usepackage{bookmark}}{\usepackage{hyperref}}
\hypersetup{
  pdftitle={Data Modeling Mindsets},
  pdfauthor={Christoph Molnar},
  hidelinks,
  pdfcreator={LaTeX via pandoc}}
\urlstyle{same} % disable monospaced font for URLs
\usepackage[margin=1in]{geometry}
\usepackage{longtable,booktabs,array}
\usepackage{calc} % for calculating minipage widths
% Correct order of tables after \paragraph or \subparagraph
\usepackage{etoolbox}
\makeatletter
\patchcmd\longtable{\par}{\if@noskipsec\mbox{}\fi\par}{}{}
\makeatother
% Allow footnotes in longtable head/foot
\IfFileExists{footnotehyper.sty}{\usepackage{footnotehyper}}{\usepackage{footnote}}
\makesavenoteenv{longtable}
\usepackage{graphicx}
\makeatletter
\def\maxwidth{\ifdim\Gin@nat@width>\linewidth\linewidth\else\Gin@nat@width\fi}
\def\maxheight{\ifdim\Gin@nat@height>\textheight\textheight\else\Gin@nat@height\fi}
\makeatother
% Scale images if necessary, so that they will not overflow the page
% margins by default, and it is still possible to overwrite the defaults
% using explicit options in \includegraphics[width, height, ...]{}
\setkeys{Gin}{width=\maxwidth,height=\maxheight,keepaspectratio}
% Set default figure placement to htbp
\makeatletter
\def\fps@figure{htbp}
\makeatother
\setlength{\emergencystretch}{3em} % prevent overfull lines
\providecommand{\tightlist}{%
  \setlength{\itemsep}{0pt}\setlength{\parskip}{0pt}}
\setcounter{secnumdepth}{5}
\setbeamercolor{structure}{fg=black}
\ifLuaTeX
  \usepackage{selnolig}  % disable illegal ligatures
\fi

\title{Data Modeling Mindsets}
\author{Christoph Molnar}
\date{}

\begin{document}
\maketitle

{
\setcounter{tocdepth}{2}
\tableofcontents
}
\hypertarget{modeling}{%
\section{Modeling}\label{modeling}}

\hypertarget{details-about-the-book}{%
\section{Details about the book}\label{details-about-the-book}}

\begin{itemize}
\tightlist
\item
  Short: 30 - 70 pages
\item
  Introduction to the topics, not to deep into the topics
\item
  Finish within a week
\item
  Publish regular manner: web, leanpub, amzn
\item
  printing format may differ
\item
  Rather inclusive of modeling mindsets
\end{itemize}

\hypertarget{preface}{%
\section{Preface}\label{preface}}

\begin{itemize}
\tightlist
\item
  When it comes to data modeling, there are very different mindsets
\item
  Bayesian stats, freq stats, causality, machine learning
\item
  I've personally met people who only follow one. Not very good.
\item
  But if you have access to multiple mindsets, you are the king
\end{itemize}

\hypertarget{data-modeling-mindsets}{%
\section{Data Modeling Mindsets}\label{data-modeling-mindsets}}

Bayesian and frequentist statistics, machine learning and causal inference -- these approaches share common methods and models.
They differ in assumptions about the data-generating process and when a model is a good generalization of the real world.

\hypertarget{machine-learning}{%
\section{Machine Learning}\label{machine-learning}}

\[\arg\min_f L(X,Y,f(X))\]

Machine learning minimizes a loss function \(L\) by finding the best function f that to predict target \(Y\) from features X.
A good machine learning model has a low loss on the test data.

\hypertarget{interpretable-machine-learning}{%
\section{Interpretable Machine Learning}\label{interpretable-machine-learning}}

It's like machine learning, but interpretability in mind.
It get's an extra mention because it differs from pure machine learning, as the focus on loss might be lessened.
Also the mindset is expanded: Not only is the best model the one that minimizes loss, but we say it also makes sense to study the model instead of the real world.

\hypertarget{statistical-inference}{%
\section{Statistical Inference}\label{statistical-inference}}

\[\arg\max_{\theta} P(\theta, X)\]

Statistical inference fits the best parameters of a chosen probability distribution for variables \(X\).
A good statistical model has a high goodness-of-fit: the data fit the distribution.

\hypertarget{bayesian-inference}{%
\section{Bayesian Inference}\label{bayesian-inference}}

\[P(\theta | X) = \frac{P(X | \theta) \cdot P(\theta)}{P(X)}\]

Bayesian inference assumes that the distribution parameters \(\theta\) are random variables with an a-priori distribution.
A good Bayesian model has a high posterior probability (Bayes factor).

\hypertarget{causal-inference}{%
\section{Causal Inference}\label{causal-inference}}

\[P(Y|do(X))\]

Causal inference operates on the principles of causality, intervention and counterfactuals..
A good causal model has high goodness-of-fit and solid causal assumptions.

\hypertarget{exploratory-modeling}{%
\section{Exploratory ``Modeling''}\label{exploratory-modeling}}

\hypertarget{expert-systems}{%
\section{Expert Systems}\label{expert-systems}}

\hypertarget{logic-programming}{%
\section{Logic Programming}\label{logic-programming}}

\hypertarget{symbolic-artificial-intelligence}{%
\section{Symbolic Artificial Intelligence}\label{symbolic-artificial-intelligence}}

\hypertarget{which-one-is-the-best}{%
\section{Which One is the Best?}\label{which-one-is-the-best}}

The smart way is to be pragmatic about the modeling choices. Need a causal interpretation? Think causal inference. Only predictive performance is important? Pick machine learning. Want to include prior information about model parameters? -\textgreater{} Bayesian stats.

\end{document}
